\documentclass[14pt]{extarticle}

% Общие пакеты
\usepackage[utf8]{inputenc}
\usepackage[T2A]{fontenc}
\usepackage[russian]{babel}

% Математические пакеты
\usepackage{amsmath, amssymb, amsthm, mathrsfs}

% Пакеты для формата страницы
\usepackage[a4paper,margin=2.5cm]{geometry}
\usepackage[notlof,notlot]{tocbibind}
\usepackage{needspace}

\usepackage{titlesec, enumitem}

% Каждая секция начинается с новой страницы
\titleformat{\section}
  {\clearpage\normalfont\Large\bfseries}{\thesection}{1em}{}

% Устанавливаем отступы
\setlength{\parindent}{0pt}

\titlespacing{\subsection}{0pt}{4\baselineskip}{1\baselineskip}
\titlespacing{\subsubsection}{0pt}{2\baselineskip}{1\baselineskip}

% Отступы для текста
\clubpenalty=10000          % Запрет одиночных строк в начале страницы
\widowpenalty=10000         % Запрет одиночных строк в конце страницы
\displaywidowpenalty=10000  % То же для формул
\brokenpenalty=10000        % Запрет переносов в неподходящих местах
\tolerance=1000             % Допустимая растяжимость строк
\emergencystretch=3pt       % Экстренная растяжимость

% Минимальное количество строк для переноса абзаца
\raggedbottom              % Позволяет страницам быть разной высоты
\setlength{\parfillskip}{0pt plus 1fil}

% Настройки для предотвращения разрывов коротких абзацев
\newcommand{\needparagraph}[1]{
    \needspace{3\baselineskip}
    #1
}

% Настройки для секций (чтобы заголовки не "висели" в одиночестве)
\titlespacing{\section}{0pt}{3\baselineskip plus 1\baselineskip}{1\baselineskip}
\titlespacing{\subsection}{0pt}{4\baselineskip}{1\baselineskip}
\titlespacing{\subsubsection}{0pt}{2\baselineskip}{1\baselineskip}

% Отступы для списков
\setlist[enumerate]{noitemsep, topsep=0pt}
\setlist[itemize]{noitemsep, topsep=0pt}

\let\olditem\item
\renewcommand{\item}{\needspace{2\baselineskip}\olditem}

% Глубина нумерации
\setcounter{secnumdepth}{3}
\setcounter{tocdepth}{3}

% Ссылки
\usepackage{hyperref, cleveref}

% Цвета для блоков
\usepackage{tcolorbox, xcolor}
\definecolor{theoremcolor}{HTML}{F0F0F0}
\definecolor{bordercolor}{HTML}{708090}
\usepackage{fancyhdr}
\pagestyle{fancy}

% Устанавливаем рекомендуемую высоту заголовка
\setlength{\headheight}{17.0pt}
\addtolength{\topmargin}{-2.5pt}

\fancyhf{} % очистить все поля

% Верхний колонтитул: название и номер секции
\fancyhead[L]{\nouppercase{\leftmark}}
\fancyhead[R]{\thepage}

% Нижний колонтитул
\fancyfoot[C]{\textit{}}

\renewcommand{\headrulewidth}{0.4pt}
\renewcommand{\footrulewidth}{0.4pt}

% Подключаем tcolorbox
\tcbuselibrary{theorems,skins,breakable}

% Стиль рамки для теорем
\tcbset{theostyle/.style={
  colback=theoremcolor!50,
  colframe=bordercolor,
  fonttitle=\bfseries,
  coltitle=black,
  boxrule=0.8pt,
  arc=2mm,
  left=1.5mm,
  right=1.5mm,
  top=1mm,
  bottom=1mm,
  enhanced,
  breakable,
  before skip=\baselineskip,
  after skip=\baselineskip
}}

% Определения новых теорем
\newtcbtheorem[auto counter, number within=section]{theorem}{Теорема}{theostyle}{th}
\newtcbtheorem[auto counter, number within=section]{lemma}{Лемма}{theostyle}{lem}
\newtcbtheorem[auto counter, number within=section]{definition}{Определение}{theostyle}{def}
\newtcbtheorem[auto counter, number within=section]{example}{Пример}{theostyle}{ex}

\title{Вычмат-2025}

\begin{document}

\maketitle
\tableofcontents

\section{Основные определения}

\subsection{Предмет вычислительной математики. Метод и задачи вычислительной математики в терминах функционального анализа.}

    \subsubsection{Предмет вычислительной математики.}

        Необходимость разработки методов доведения математических исследований до числового результата привела к созданию отдельной дисциплицы - \textbf{вычислительной математики}.

        \begin{definition}{Вычислительная математика-1}{def-compmath-1}
            Область математики, которая призвана разрабатывать методы доведения до числового результата решений основных задач математического анализа, алгебры и геометрии и пути использования для этой цели современных вычислительных средств.
        \end{definition}

        \begin{definition}{Вычислительная математика-2}{def-compmath-2}
            Раздел математики, связанный с построением и анализом алгоритмов численного решения математических задач.
        \end{definition}

        Таким образом, \textbf{вычислительная математика} помогает решать численные задачи с помощью ЭВМ.

    \subsubsection{Функциональный анализ.}
        \begin{definition}{Функциональный анализ}{def-func-analysis}
            Область математики, изучающая свойства функциональных пространств.
        \end{definition}

        Для определения \textbf{задач и методов} вычислительной математики введем важнейшие \textbf{понятия функционального анализа}.

        \begin{definition}{Понятия функционального анализа}{def-func-analysis}
            \begin{itemize}
                \item Функциональные метрические пространства.
                \item Функции, определенные на функциональных пространствах.
            \end{itemize}
        \end{definition}

        \textbf{Функциональный анализ} рассматирвает элементы более общего (не евклидова) пространства.

    \subsubsection{Функциональные метрические пространства.}
        
        В функциональном анализе вместо евклидовых пространств рассматриваются абстрактные пространства, элементы которых могут иметь самую различную природу.

        \begin{definition}{Метрическое пространство}{def-metric-space}
            Абстрактное множество, для любых двух элементов $x$ и $y$ которого опрделено понятие расстояния $\rho(x, y)$.
        \end{definition}

        \begin{lemma}{Свойства расстояния}{lem-distance-features}
            Расстояние $\rho(x, y)$ должно удовлетворять следующим \textbf{свойствам:}
            \begin{enumerate}
                \item $\rho(x,y) \geq 0$, причем $\rho(x, y) = 0$ $\leftrightarrow$ $x$ совпадает с $y$.
                \item $\rho(x, y) = \rho(y, x)$.
                \item $\rho(x, y) \leq \rho(x, z) + \rho(z, y)$ $\forall$ $x, y, z \in \mathscr{R}$, где: $\mathscr{R}$ - метрическое пространство.
            \end{enumerate}
        \end{lemma}

        Евклидовы пространства с обычным определением расстояния удовлетворяют всем этим условиям. Но могут быть и другие метрические пространства.

        \begin{definition}{Пространство непрерывных функций}{def-continuous-func-space}
            Пространство $C[a, b]$ - множество всех непрерывных функций на отрезке $[a, b]$.\\
            Функция $f(x)$ непрерывная на $[a, b]$ $\leftrightarrow$ $f(x) \in C[a, b]$.
        \end{definition}

        \begin{example}{Неевклидово метрическое пространство}{ex-noneuclidean-metric-space}
            Пространства $L_{p}$, где $p \geq 1$ и $p \in \mathbb{R}$.
            $$L_{p} = \{f(x) \text{| } f(x) \in \text{C}[a, b] \text{, } \int_{a}^{b}|f(t)|^{p} \, dt < \infty\}$$

            \vspace{\baselineskip}

            Расстояние $\rho(x, y)$ в пространстве $L_{p}$ определяется следующим образом:
            $$\rho(x, y) = [\int_{a}^{b}|x(t) - y(t)|^{p}]^{\frac{1}{p}}$$
        \end{example}

        В каждом метрическом пространстве можно говорить об \textbf{окрестности данной точки}.

        \begin{definition}{Окрестность точки}{def-point-locality}
            $\varepsilon$-окрестностью точки $x$ некоторого метрического пространства $\mathscr{R}$ называется множество точек $y$ таких, что:
            $$\rho(x, y) \leq \varepsilon$$
        \end{definition}

        \begin{example}{Окрестность точки в $L_{p}$}{ex-point-locality}
            Окрестность точки в $L_{p}$ - это совокупность всех функций $y(t)$, принадлежащих $L_{p}$, для которых:
            $$\int_{a}^{b} |x(t) - y(t)|^{p} \, dt < \varepsilon^{p}$$
        \end{example}

        В вычислительной математике часто приходится заменять одну функцию $x(t)$ другой, более удобной для вычислительных целей и в каком-то смысле близкой к первой. Обычно эту вторую функцию берут из $\varepsilon$-окрестности первой.

    \subsubsection{Функции, заданные на функциональном пространстве.}
        
        \begin{definition}{Операторы функционального пространства}{def-operator}
            Пусть нам даны два абстрактных (функциональных) пространства $\mathscr{R}_{1}$ и $\mathscr{R}_{2}$. Пусть каждому элементу $x \in \mathscr{R}_{1}$ поставлен в соответствие элемент $y \in \mathscr{R}_{2}$. Тогда будем говорить, что нам задан \textbf{оператор}:\\
            $$y = A(x)$$
            с областью определения $\mathscr{R}_{1}$ и областью значений, принадлежащих $\mathscr{R}_{2}$.

            \vspace{\baselineskip}

            В частности, если $\mathscr{R}_{2}$ является областью вещественных или комплексных чисел, то оператор $A(x)$ - \textbf{функционал}.
        \end{definition}

        \begin{example}{Функционал}{ex-operator}
            Оператором (функционалом) в пространстве непрерывных функций на отрезке $[a, b]$ $C[a, b]$ - \textbf{определенный интеграл}:
            $$I(x) = \int_{a}^{b} x(t) \, dt$$
        \end{example}

    \subsubsection{Методы и задачи вычислительной математики.}

        \begin{definition}{Задачи вычислительной математики}{def-compmath-problems}
            Многие задачи в вычислительной математике могут быть записаны в виде:
            $$y = A(x)$$
            где $x$ и $y$ принадлежат заданным пространствам $\mathscr{R}_{1}$ и $\mathscr{R}_{2}$ и $A(x)$ - некоторых заданный оператор.
        \end{definition}

        Далеко не всегда с помощью средств современной математики удается точно решить эти задачи, применяя конечное число шагов. Для этого используют \textbf{методы вычислительной математики}:

        \begin{definition}{Основной метод вычислительной математики}{def-compmath-main-method}
            Замена пространств $\mathscr{R}_{1}$ и $\mathscr{R}_{1}$ и $\mathscr{R}_{2}$ и оператора $A(x)$ другими пространствами $\overline{\mathscr{R}_{1}}$ $\overline{\mathscr{R}_{2}}$ и оператором $\overline{A}$, более удобными для вичислительных целей.

            \vspace{\baselineskip}

            Иногда бывает достаточно произвести замену только пространств $\mathbb{R}_{1}$ и $\mathbb{R}_{2}$ или даже одного из них. Иногда достаточно заменить только оператор.

            \vspace{\baselineskip}

            Замена должна быть сделана так, чтобы решение новой задачи
            $$\overline{y} = \overline{A(\overline{x})}$$
            $\overline{x} \in \overline{\mathscr{R_{1}}}$, $\overline{y} \in \overline{\mathscr{R_{2}}}$, было в каком-то смысле близким к точному значению исходной задачи и его можно было бы отыскать сравнительно небольшими трудностями.\\
            Т.е.:
            $$\rho(x, \overline{x}) < \varepsilon$$
            $$\rho(y, \overline{y}) < \varepsilon$$
        \end{definition}

        \begin{example}{Применение метода}{ex-method-application}
            $f(x) \in C[a, b]$. Требуется решить задачу:
            $$y = \int_{a}^{b} f(x) \, dx$$
            причем интеграл не берется в элементарных функциях.

            \vspace{\baselineskip}

            Тогда возможны два пути:
            \begin{enumerate}
                \item \textbf{Замена пространств:} вместо $f(x)$ взять $P_{n}(x)$ - алгебраический многочлен степени $n$.
                \item \textbf{Замена оператора:} вместо интегрирования построить интегральную сумму $\sum_{i=1}^{n}f(x_{i})\Delta_{i}$.
            \end{enumerate}
        \end{example}

        \begin{definition}{Основные вычислительные методы}{def-main-comp-methods}
            \textbf{Вычислительные методы} - методы, что используются в вычислительной математике для преобразования задач к виду, удобному для реализации на ЭВМ.

            \vspace{\baselineskip}

            Основные классы вычислительных методов:
            \begin{itemize}
                \item Методы эквивалентных преобразований (замена исходной задачи другой (более простой), имеющее то же решение).
                \item Методы аппроксимации (аппроксимировать исходную задачу другой, решение которой в определенном смысле близко к решению исходной задачи).
                \item Итерационные методы (для получения приближений выполняют однотипный набор действий с использований найденных ранее приближений - итерацию).
            \end{itemize}

        \end{definition}

        Резюмируя, можно выделить \textbf{основные задачи} вычислительной математики:

        \begin{example}{Основные задачи}{ex-main-compmath-problems}
            \begin{itemize}
                \item Приближение множеств в функциональных пространствах.
                \item Приближение операторов, заданных на функциональных пространствах.
                \item Разработка рациональных алгоритмов и методов решения задач в условиях приминения современных вычислительных средств.
            \end{itemize}
        \end{example}

\clearpage
\subsection{Источники и классификация погрешностей результатов численного решения задач. Приближенные числа. Абсолютная и относительная погрешности. Правила записи приближенных чисел.}

    \subsubsection{Источники и классификация погрешностей}

        При решении прикладной задачи с использованием ЭВМ получить точное решение задачи практически невозможно. Получаемое решение почти всегда содержит погрешность, т.е. является приближенным. 
        \begin{definition}{Источники погрешности решения}{def-error-sources}
            Пусть y - точное значение величины, вычисление которой является целью поставленной задачи, а $y^{*}$ - ее приближенное значение:
            
            \begin{enumerate}
                \item \textbf{Неустранимая погрешность:} $\delta_{\text{н}}y^{*}$ - математическая модель и исходные данные вносят в решение ошибку, которая не может быть устранена далее.
                \item \textbf{Ошибка метода решения:} $\delta_{\text{м}}y^{*}$ - источник данной погрешности - метод решения задачи.
                \item \textbf{Вычислительная погрешность:} $\delta_{\text{в}}y^{*}$ - определяется характеристикой машины ЭВМ.
            \end{enumerate}
        \end{definition}

        Таким образом, полная погрешность результата решения задачи на ЭВМ складывается из трех составляющих:
        $$\delta y^{*} = \delta_{\text{н}}y^{*} + \delta_{\text{м}}y^{*} + \delta_{\text{в}}y^{*}$$

        \vspace{\baselineskip}

        На практике исходят из того, что погрешность метода должна быть на порядок (в 2 - 10 раз) меньше неустранимой погрешности. Желательно, чтобы величина вычислительной ошибки была хотя бы на порядок меньше величины погрешности метода.

    \subsubsection{Приближенные числа. Абсолютная и относительная погрешности.}
        
        Пусть $а$ - точное (неизвестное) значение некоторой величины, $a^{*}$ - приближенное (известное) значение той же величины (приближенное число).

        \begin{definition}{Абсолютная погрешность}{def-absolute-error}
            Модуль разности приближенного и точного значения некоторой величины:
            $$\Delta(a^{*}) = |a - a^{*}|$$
        \end{definition}

        \begin{definition}{Относительная погрешность}{def-relative-error}
            Для соотншения погрешность величины и ее значения вводят понятие \textbf{относительной погрешности}:
            $$\delta(a^{*}) = \frac{|a - a^{*}|}{|a|} = \frac{\Delta(a^{*})}{|a|}$$
        \end{definition}

        Так как значение $а$ неизвестно, то непосредственное вычисление величин $\Delta(a^{*})$ и $\delta(a^{*})$ по предыдущим формулам невозможно.

        \begin{definition}{Верхние границы погрешностей}{def-upper-limits-error}
            $\overline{\Delta(a^{*})}$ и $\overline{\delta(a^{*})}$ - верхние границы абсолютной и относительной погрешностей соответственно:
            $$|a - a^{*}| \leq \overline{\Delta(a^{*})}$$
            $$\frac{|a - a^{*}|}{|a|} \leq \overline{\delta(a^{*})}$$

            \vspace{\baselineskip}

            Причем, если величина $\overline{\Delta(a^{*})}$ известна, то:
            $$\overline{\delta(a^{*})} = \frac{\overline{\Delta(a^{*})}}{|a|}$$
        
            \vspace{\baselineskip}

            Аналогично, если известна $\overline{\delta(a^{*})}$:
            $$\overline{\Delta(a^{*})} = |a| \cdot \overline{\delta(a^{*})}$$            
        \end{definition}

    \subsubsection{Правила записи приближенных чисел.}
        
        Пусть приближенное число $a^{*}$ задано следующим образом:
        $$a^{*} = \alpha_{n}\alpha_{n-1}\ldots\alpha_{0}.\beta_{1}\beta_{2}\ldots\beta_{m}$$
        где $\alpha_{n}\alpha_{n-1}\ldots\alpha_{0}$ - целая часть, $\beta_{1}\beta_{2}\ldots\beta_{m}$ - дробная.

        \begin{definition}{Значащие цифры}{def-significant-digits}
            Все цифры в записи числа $a^{*}$, начиная с первой ненулевой слева.
        \end{definition}

        \begin{definition}{Верная цифра}{def-correct-digit}
            Значащую цифру называют \textbf{верной}, если абсолютная погрешность числа не превосходит единицы разряда, соответствующей этой цифре.
        \end{definition}

        \begin{example}{Значащие и верные цифры}{ex-significant-and-correct-digits}
            Пусть $a^{*} = 0.010300$, $\Delta(a^{*}) = 2 \cdot 10^{-6}$:
            
            \begin{enumerate}
                \item \textbf{Значащие цифры:} $10300$
                \item \textbf{Верные цифры:} $1030$
            \end{enumerate}
        \end{example}

        \begin{lemma}{Связь числа верных цифр с отностительной погрешностью}{lem-relative-correct-connection}
            Если число $a^{*}$ имеет ровно $N$ верных цифр, то $\delta(a^{*}) \sim 10^{-N}$.
        \end{lemma}

        \begin{lemma}{Правило записи}{lem-note-rule}
            Неравенство верхней границы абслолютной погрешности эквивалентно следующему:
            $$a^{*} - \overline{\Delta{a^{*}}} \leq a \leq a^{*} + \overline{\Delta{a^{*}}}$$
            
            Тот факт, что число $a^{*}$ является приближенным значением числа $a$ с абслоютной точностью $\varepsilon = \overline{\Delta(a^{*})}$ принято записывать в виде:
            $$a = a^{*} \pm \overline{\Delta(a^{*})}$$
            Как правило, числа $a^{*}$ и $\overline{\Delta(a^{*})}$ указывают с одинаковым числом цифр после десятичной точки.
        
            \vspace{\baselineskip}
        
            Аналогично из неравенства верхней границы относительной погрешности получаем:
            $$a^{*}(a - \overline{\delta{a^{*}}}) \leq a \leq a^{*}(a + \overline{\delta{a^{*}}})$$
            
            Тот факт, что число $a^{*}$ является приближенным значением числа $a$ с относительной точностью $\varepsilon = \overline{\delta(a^{*})}$ принято записывать в виде:
            $$a = a^{*}(1 \pm \overline{\delta(a^{*})})$$
        \end{lemma}

    Если число $a^{*}$ приводится в качестве результата \textbf{без указания величины погрешности}, то принято считать, что все его значащие цифры являются \textbf{верными}.


    \subsubsection{Округления.}

    \begin{definition}{Округление методом усечения}{def-truncation-rounding}
        Состоит в отбрасывании всех цифр, расположенных слева от $n$-ой значащей цифры.
    \end{definition}

    \begin{definition}{Округление по дополнению}{def-addition-rounding}
        Состоит в следующем правиле: если первая слева от отбрасываемых цифр меньше $5$, то сохраняемые цифры остаются без изменения. Иначе: в младший сохраняемый разряд добавляется единица.  
    \end{definition}

    Границы абсолютной и относительной погрешностей принято округлять в сторону увеличения.

\clearpage
\subsection{Погрешности арифметических операций над приближенными числами. Погрешность функции одной и многих переменных.}

    \subsubsection{Погрешности арифметических операций над приближенными числами.}
        
        \begin{theorem}{Абсолютная погрешность сложения/вычитания}{th-absolute-error-addition-subtraction}
            Абсолютная погрешность алгебраической суммы или разности не превосходит суммы абсолютных погрешностей слагаемых, т.е.:
            $$\Delta(a^{*} \pm b^{*}) \leq \Delta(a^{*}) + \Delta(b^{*})$$
        
            \begin{proof}
                $\Delta(a^{*} \pm b^{*}) = |(a \pm b) - (a^{*} \pm b^{*})| = |(a - a^{*} \pm (b - b^{*}))| \leq \Delta(a^{*}) + \Delta(b^{*})$\\
            \end{proof}
        \end{theorem}

        \begin{consequence}{Абсолютная погрешность сложения/вычитания}{con-absolute-error-addition-subtraction}
            В силу того, что $\Delta(a^{*}) \leq \overline{\Delta(a^{*})}$, получаем:
            $$\overline{\Delta(a^{*} \pm b^{*})} = \overline{\Delta(a^{*})} + \overline{\Delta(b^{*})}$$
        \end{consequence}

        \begin{theorem}{Относительная погрешность сложения/вычитания}{th-relative-error-addition-subtraction}
            Пусть a и b ненулевые числа одного знака. Тогда справедливы неравенства:
            $$\delta(a^{*} + b^{*}) \leq \delta_{\max} \text{, } \delta(a^{*} - b^{*}) \leq \nu \delta_{\max}$$
            где: $\delta_{\max} = \max\{\delta(a^{*}) \text{, } \delta(b^{*})\}$, $\nu = \frac{|a + b|}{|a - b|}$
        
            \vspace{\baselineskip}

            \begin{proof}
                $|a + b|\delta(a^{*} + b^{*}) = \Delta(a^{*} + b^{*}) \leq \Delta(a^{*}) + \Delta(b^{*}) = |a|\delta(a^{*}) + |b|\delta(b^{*}) \leq |a|\delta_{\max} + |b|\delta_{\max} = (|a| + |b|)\delta_{\max} = |a + b|\delta_{\max}$ (последний переход - равенство, т.к. числа одного знака)\\
                Т.е. $\delta(a^{*} + b^{*}) \leq \delta_{\max}$

                \vspace{\baselineskip}

                $|a - b|\delta(a^{*} - b^{*}) = \Delta(a^{*} - b^{*}) \leq \Delta(a^{*} + \Delta(b^{*})) \leq |a + b| \delta_{\max}$\\
                Т.е. $\delta(a^{*} - b^{*}) \leq \frac{|a + b|}{|a -b|}\delta_{\max} = \nu \delta_{\max}$
            \end{proof}
        \end{theorem}

        Итог: При построении численного метода решения задачи следует избегать вычитания близких чисел одного знака. Если же такое вычитание неизбежно, то следует вычислять  аргументы  с  повышенной  точностью,  учитывая  ее  потерю примерно в $\nu = \frac{|a+b|}{|a-b|}$ раз.

        \begin{theorem}{Относительная погрешность умножения/деления}{th-relative-error-mul-div}
            Для относительных погрешностей произведения и частного приближенных чисел верны оценки:
            $$\delta(a^{*}b^{*}) \leq \delta(a^{*}) + \delta(b^{*}) + \delta(a^{*})\delta(b^{*})$$
            $$\delta(\frac{a^{*}}{b^{*}}) \leq \frac{\delta(a^{*}) + \delta(b^{*})}{1 - \delta(b^{*})}$$

            \begin{proof}
                $|ab|\delta(a^{*}b^{*}) = \Delta(a^{*}b^{*}) = \Delta(a^{*}b^{*}) = |ab - a^{*}b^{*}| = |(a - a^{*})b + (b - b^{*})a - (a - a^{*})(b - b^{*})| \leq |a - a^{*}| \cdot |b| + |b - b^{*}| \cdot |a| + |a - a^{*}| \cdot |b - b^{*}| = \Delta(a^{*})|b| + \Delta(b^{*})|a| + \Delta(a^{*})\Delta(b^{*}) = c$

                \vspace{\baselineskip}

                Разделим $c$ на $|ab|$:\\
                $\frac{c}{|ab|} = \delta(a^{*}) + \delta(b^{*}) + \delta(a^{*})\delta(b^{*})$
            
                \vspace{\baselineskip}

                Итог: $\delta(a^{*}b^{*}) \leq \delta(a^{*}) + \delta(b^{*}) + \delta(a^{*})\delta(b^{*})$

                \vspace{\baselineskip}

                $|\frac{a}{b}|\delta(\frac{a^{*}}{b^{*}}) = \Delta(\frac{a^{*}}{b^{*}}) = |\frac{a}{b} - \frac{a^{*}}{b^{*}}| = |\frac{ab^{*} - a^{*}b}{bb^{*}}| = c$

                \vspace{\baselineskip}

                $|b^{*}| = |b - (b - b^{*})| = |b| \cdot |1 - \frac{b - b^{*}}{b}| \geq |b| \cdot (1 - \delta(b^{*}))$

                \vspace{\baselineskip}

                $c \leq \frac{|ab^{*} - a^{*}b|}{|b|^{2}(1 - \delta(b^{*}))}$

                \vspace{\baselineskip}

                Разделим на $|\frac{a}{b}|$:\\
                $\delta(\frac{a^{*}}{b^{*}}) \leq \frac{\delta(a^{*} + b^{*})}{1 - \delta(b^{*})}$
                
                \vspace{\baselineskip}

                Итог: $\delta(\frac{a^{*}}{b^{*}}) \leq \frac{\delta(a^{*}) + \delta(b^{*})}{1 - \delta(b^{*})}$
            \end{proof}
        \end{theorem}

        \begin{consequence}{Относительная погрешность умножения/деления}{con-relative-error-mul-div}
            Если $\delta(a^{*}) << 1$ и $\delta(b^{*}) << 1$, то для оценки границ относительных погрешностей можно использовать следующие приближенные равенства:
            $$\overline{\delta(a^{*}b^{*})} \approx \overline{\delta(a^{*})} + \overline{\delta(b^{*})}$$
            $$\overline{\delta(\frac{a^{*}}{b^{*}})} \approx \overline{\delta(a^{*})} + \overline{\delta(b^{*})}$$
        \end{consequence}

        \textbf{Общий итог:} выполнение арифметических операций над приближенными числами, как правило, сопровождается потерей точности. Единственная операция, при которой потеря не происходит, это сложение чисел одного знака. Наибольшая потеря точности может произойти при вычитании близких чисел одного знака.

    \subsubsection{Погрешность функции одной и многих переменной.}

        \begin{theorem}{Погрешность функции одной переменной}{th-error-function-one-variable}
            Формулы для границ погрешностей функции одной переменной:
            $$\overline{\Delta(y^{*})} \approx |f^{'}(x^{*})|\overline{\Delta(x^{*})}$$
            $$\overline{\delta(y^{*})} \approx \nu^{*}\overline{\delta(x^{*})}$$
            $$\overline{\delta(y^{*})} \approx \nu \overline{\delta(x^{*})}$$
        
            где $\nu^{*} = |x^{*}| \frac{f^{'}(x^{*})}{f(x^{*})}$, $\nu = |x| \frac{f^{'}(x)}{f(x)}$
       
            \begin{proof}
                Частный случай формул погрешностей функции многих переменных.
            \end{proof}
        \end{theorem}

        \begin{theorem}{Погрешность функции многих переменных}{th-error-function-many-variables}
            Пусть $f(\vec{x}) = f(x_{1}, x_{2}, \ldots, x_{m})$ - дифференцируемая в области $G$ функция $m$ переменных, вычисление которой производится при приближенно заданных аргументах $x_{1}^{*}, x_{2}^{*}, \ldots, x_{m}^{*}$. Тогда:
            $$\Delta(y^{*}) \leq \sum_{j=1}^{m} \max_{[x, x^{*}]} |f_{x_{j}}^{'}|\Delta(x_{j}^{*})$$
        
            \begin{proof}
                Вытекает из формулы конечных приращений Лагранжа:\\
                $$f(\vec{x}) - f(\vec{x^{*}}) = \sum_{j=1}^{m} f_{x_{j}}^{'}(\overline{x})(x_{j} - x_{j}^{*}) \text{, } \overline{x} \in [x, x^{*}]$$
                
                Далее берем модуль от правой и левой частей уравнения и правую часть заменяем на максимум. Получаем требуемое соотношение.
            \end{proof}
        \end{theorem}

        \begin{consequence}{Погрешность функции многих переменных}{con-error-function-many-variables}
            Если $x^{*} \approx x$, то можно положить:
            $$\overline{\Delta(y^{*})} \approx \sum_{j=1}^{m}|f_{x_{j}}^{'}(x)|\overline{\Delta(x_{j}^{*})}$$
            $$\overline{\Delta(y^{*})} \approx \sum_{j=1}^{m}|f_{x_{j}}^{'}(x^{*})|\overline{\Delta(x_{j}^{*})}$$
        
            \vspace{\baselineskip}
        
            Из этих формул вытекают приближенные равенства для оценки границ относительных погрешностей:
            $$\overline{\delta(y^{*})} \approx \sum_{j=1}^{m} \nu_{j}\overline{\delta(x_{j}^{*})}$$
            $$\overline{\delta(y^{*})} \approx \sum_{j=1}^{m} \nu_{j}^{*}\overline{\delta(x_{j}^{*})}$$
        
            где:
            $$\nu_{j} = \frac{|x_{j}|\cdot|f_{x_{j}}^{'}(x)|}{|f(x)|} \text{, } \nu_{j} = \frac{|x_{j}^{*}|\cdot|f_{x_{j}}^{'}(x^{*})|}{|f(x^{*})|}$$
        \end{consequence}

\clearpage
\subsection{Корректность вычислительной задачи. Примеры корректных и некорректных задач.}
    
    \begin{definition}{Вычислительная задача}{def-comp-task}
        \textbf{Постановка} вычислительной задачи \textbf{включает в себя}: 
        \begin{enumerate}
            \item \textbf{Задание} множества допустимых входных данных $X$.
            \item \textbf{Задание} множества возможных решений $Y$. 
        \end{enumerate}

        \vspace{\baselineskip}

        \textbf{Цель} вычислительной задачи состоит в нахождении решения $y \in Y$ по заданному входному данному $x \in X$.
    \end{definition}

    \begin{definition}{Корректность вычислительной задачи}{def-comp-task-correctness}
        Вычислительная задача называется \textbf{корректной}, если выполнены следующие три требования: 
        \begin{enumerate}
            \item Решение $y \in Y$ \textbf{существует} при любых входных данных $x \in X$.
            \item Решение \textbf{единственно}.
            \item Решение \textbf{устойчиво} по отношению к малым возмущениям входных данных (решение зависит от входных данных непрерывным образом: $\forall \varepsilon > 0 \text{ } \exists \delta = \delta(\varepsilon) > 0 \text{: } \forall x^{*} \text{: } \Delta{x^{*}} < \delta \rightarrow y^{*} \text{: } \Delta(y^{*}) < \varepsilon$). 
        \end{enumerate}

        \vspace{\baselineskip}

        В том случае, когда \textbf{хотя бы одно} из этих требований \textbf{не выполнено}, задача называется \textbf{некорректной}.  
    \end{definition}

    \begin{example}{Корректная вычислительная задача}{ex-correct-comp-task}
        Решение квадратного уравнения: $x^{2} + bx + c = 0$ ($a = 1$).
        $$x_{1, 2} = \frac{-b \pm \sqrt{b^{2} - 4c}}{2}$$
        \begin{itemize}
            \item \textbf{Наличие решения:} в области $\mathbb{R}$ должно выполняться неравенство: $b^{2} - 4ac \geq 0$.
            \item \textbf{Единственность решения:} два корня можно представить в виде вектора $\begin{pmatrix} x_{1} \\ x_{2} \end{pmatrix}$.
            \item \textbf{Устойчивость решения:} корни являются непрерывными функциями коэффициентов $b$ и $c$.
        \end{itemize}

        \vspace{\baselineskip}

        Вычисление определенного интеграла: $I = \int_{a}^{b} f(x) \, dx$ ($f(x) \in C[a, b]$).\\
        Пусть $I^{*} = \int_{a}^{b} f^{*}(x) \, dx$, $\Delta(f^{*}(x)) = \max_{x \in [a, b]}|f(x) - f^{*}(x)|$. Тогда:\\
        $\Delta(I^{*}) = |I - I^{*}| = |\int_{a}^{b} f(x) \, dx - \int_{a}^{b} f^{*}(x) \, dx| \leq \int_{a}^{b} |f(x) - f^{*}(x)| \, dx \leq (b - a) \cdot \Delta(f^{*}(x))$\\
        Значит, $\forall \varepsilon > 0$ неравенство $\Delta(I^{*}) < \varepsilon$ будет выполено, если потребовать выполнения условия $\Delta(f^{*}(x)) < \delta = \frac{\varepsilon}{b - a}$.
    \end{example}

    \begin{example}{Некорректная вычислительная задача}{ex-incorrect-comp-task}
        Нахождение ранга матрицы в общем случае: $A \in M_{n}(R)$\\
        Пусть $A = \begin{pmatrix} 1 & 0 \\ 0 & 0 \end{pmatrix}$, $A_{\varepsilon} = \begin{pmatrix} 1 & 0 \\ 0 & \varepsilon \end{pmatrix}$. Тогда:
        $$rk(A) = 1 \text{, } rk(A_{\varepsilon}) = 2$$
        Т.е. задача неустойчива.

        \vspace{\baselineskip}

        Вычисление производной $u(x) = f^{'}(x)$ приближенно заданной функции.\\
        Пусть $f \in C^{1}[a, b]$, $f^{*}(x)$ - приближенная функция, $u^{*}(x) = (f^{*})^{'}(x)$. Тогда:
        $$\Delta(f^{*}(x)) = \max_{x \in [a, b]}|f(x) - f^{*}(x)|$$
        $$\Delta(u^{*}(x)) = \max_{x \in [a, b]}|u(x) - u^{*}(x)|$$
        
        Если взять $f^{*}(x) = f(x) + \alpha \sin(\frac{x}{\alpha^{2}})$, где $0 < alpha << 1$. Тогда:
        $$u^{*}(x) = u(x) + \alpha^{-1}\cos(\frac{x}{\alpha^{2}})$$
        
        Следовательно:
        $$\Delta(u^{*}) = \alpha^{-1} \text{, } \Delta(f^{*}) = \alpha$$

        Значит, сколь угодно малой погрешности задания функции $f(x)$ может отвечать сколь угодно большая погрешность производной $f^{'}(x)$.
    \end{example}

\clearpage
\subsection{Обусловленность вычислительной задачи. Примеры хорошо и плохо обусловленных задач.}

    На пракстике погрешность исходных данных не всегда сколь угодно малая, точность их ограничена.

    \begin{definition}{Обусловленность вычислительной задачи}{def-conditionaly-comp-task}
        Чувствительность решения задачи к малым погрешностям исходных данных.

        \vspace{\baselineskip}

        Задачу называют: 
        \begin{itemize}
            \item \textbf{хорошо обусловленной}, если малым погрешностям исходных данных отвечают малые погрешности решения. 
            \item \textbf{плохо обусловленной}, если возможны сильные изменения решения при малых погрешностях исходных данных.
        \end{itemize}
    \end{definition}

    \begin{definition}{Число обусловленности}{def-conditionality-number}
        Мера степени обусловленности вычислительной задачи. Эту величину можно интерпретировать как коэффициент возможного возрастания погрешностей в решении по отношению к вызвавшим их погрешностям входных данных.
   
        \vspace{\baselineskip}
   
        Обычно под числом обусловленности понимают одну из величин ($\nu_{\Delta}$, $\nu_{\delta}$):
        \begin{itemize}
            \item \textbf{Абсолютное число обусловленности} ($\nu_{\Delta}$): $\Delta(y^{*}) \leq \nu_{\Delta}\Delta(x^{*})$.
            \item \textbf{Относительное число обусловленности} ($\nu_{\delta}$): $\delta(y^{*}) \leq \nu_{\delta}\delta(x^{*})$.
        \end{itemize}
    \end{definition}

    Для плохо обусловленной задачи $\nu >> 1$. Если $\nu_{\delta} \approx 10^{N}$, то порядок $N$ показывает число верных цифр, которое может быть утеряно в результате по сравнению с числом верных цифр входных данных.

    \begin{definition}{Обусловленность задачи вычисления функции одной переменной}{def-conditionality-of-func-one-variable}
        Для задачи, состоящей в вычислении по заданному $x$ значения $y = f(x)$ дифференцируемой функции $f(x)$, числа обусловленности примут вид:
        $$\nu_{\Delta} \approx |f^{'}(x)|$$
        $$\nu_{\delta} \approx \frac{|x| \cdot |f^{'}(x)|}{|f(x)|}$$
    \end{definition}

    \begin{example}{Обусловленность вычислительных задач}{ex-conditionality-comp-task}
        Задача вычисления значения функции $y = \exp^{x}$:\\
        Относительное число обусловленности: $\nu_{\delta} = |x|$. При реальных вычислениях эта величина не может быть очень большой (в противном случае переполнение).\\
        Следовательно, задача вычисления значения этой функции хорошо обусловлена, однако в случае $10 < |x| < 10^{2}$ следует ожидать потери $1$-$2$ верных значащих цифр по сравнению с числом верных цифр аргумента $x$.

        \vspace{\baselineskip}

        Задача вычисления значения функции $y = \sin(x)$:\\
        $\nu_{\Delta} = |\cos(x)| \leq 1$, $\nu_{\delta} = |\cot(x)| \cdot |x|$.\\
        При $x \rightarrow \pi k$ $\nu_{\delta} \rightarrow \infty$. Следовательно, задача плохо обусловлена.

        \vspace{\baselineskip}

        Задача вычисления определенного интеграла: $I = \int_{a}^{b} f(x) \, dx$.\\
        $\Delta(I^{*}) = |I - I^{*}| = |\int_{a}^{b} f(x) - f^{*}(x) \, dx| \leq \int_{a}^{b} |f(x) - f^{*}(x)| \, dx$\\
        $\delta(I^{*}) \leq \frac{\int_{a}^{b} |f(x) - f^{*}(x)| \, dx}{|\int_{a}^{b} f(x) \, dx|} \leq \frac{\int_{a}^{b} |\frac{f(x) - f^{*}(x)}{f(x)}| \cdot |f(x)| \, dx}{|\int_{a}^{b} f(x) \, dx|} \leq \frac{\int_{a}^{b} \delta(f^{*}(x)) |f(x)| \, dx}{|\int_{a}^{b} f(x) \, dx|} \leq \frac{\int_{a}^{b} |f(x)| \, dx}{|\int_{a}^{b} f(x) \, dx|} \cdot \overline{\delta(x)}$.\\

        Таким образом, $\delta(I^{*}) \leq \frac{\int_{a}^{b} |f(x)| \, dx}{|\int_{a}^{b} f(x) \, dx|} \cdot \overline{\delta(x)}$.\\
        Значит, при знакопостоянной функции $f(x)$, $\nu_{\delta} \approx 1$. Иначе: $\nu_{\delta} > 1$ (если $f(x)$ сильно осцилированная).
    \end{example}

\clearpage
\subsection{Вычислительные алгоритмы. Корректность и обусловленность вычислительных алгоритмов.}

    \begin{definition}{Вычислительный алгоритм}{def-comp-algorithms}
        Вычислительный метод, доведенный до степени детализации (точное предписание действий), позволяющей реализовать его на ЭВМ.
    \end{definition}

    \begin{definition}{Корректность вычислительных алгоритмов}{def-correctness-comp-algorithms}
        Вычислительный алгоритм - корректный, если выполнены условия:
        \begin{itemize}
            \item Алгоритм за конечное число элементарных для ЭВМ операций (сложение, вычитание, умножение, деление) приводит к достижению результата.
            \item Алгоритм устойчив по отношению к малым погрешностям исходных данных.
            \item Алгоритм вычислительно устойчив, т.е.: погрешность решения стремится к нулю, если машинный эпсилон стремится к нулю.
        \end{itemize}
    \end{definition}

    \begin{definition}{Обусловленность вычислительных алгоритмов}{def-conditionality-comp-algorithms}
        Отражает чувствительность результата работы алгоритма к малым, но неизбежным ошибкам округления.
   
        Алгоритм называют:
        \begin{itemize}
            \item \textbf{хорошо обусловленным}, если малые относительные погрешности округления (характеризуемые машинной точностью $\varepsilon_{М}$) приводят к малой относительной вычислительной погрешности $\delta(y^{*})$ результата $y^{*}$
            \item \textbf{плохо обусловленным}, если вычислительная погрешность может быть недопустимо большой.
        \end{itemize}
   
    \end{definition}

    \begin{definition}{Число обусловленности вычислительного алгоритма}{def-conditionality-number-alg}
        Если $\delta(y^{*})$ и $\varepsilon_{М}$ связаны неравенством $\delta(y^{*}) \leq \nu_{А}\varepsilon_{М}$, то число $\nu_{А}$ называют \textbf{числом обусловленности} вычислительного алгоритма.\\
        Для плохо обусловленных алгоритмов $\nu_{А} >> 1$.
    \end{definition}

\section{Решение нелинейных уравнений, СЛАУ}

\subsection{Постановка задачи решения нелинейных уравнений. Основные этапы решения задачи.}

\subsection{Скорость сходимости итерационных методов уточнения решения нелинейного уравнения.}

\subsection{Обусловленность задачи решения нелинейных уравнений. Понятие об интервале неопределенности. Правило Гарвика.}

\section{Интерполяция}


\section{Дифференцирование и интегрирование}

\section{Список вопросов}
\begin{enumerate}
    \item Предмет вычислительной математики. Метод и задачи вычислительной математики в терминах функционального анализа. 
    \item Источники и классификация погрешностей результатов численного решения задач. Приближенные числа. Абсолютная и относительная погрешности. Правила записи приближенных чисел. 
    \item Погрешности арифметических операций над приближенными числами. Погрешность функции одной и многих переменных. 
    \item Корректность вычислительной задачи. Примеры корректных и некорректных задач. 
    \item Обусловленность вычислительной задачи. Примеры хорошо и плохо обусловленных задач. 
    \item Вычислительные алгоритмы. Корректность и обусловленность вычислительных алгоритмов. 
    \item Постановка задачи решения нелинейных уравнений. Основные этапы решения задачи. 
    \item Скорость сходимости итерационных методов уточнения решения нелинейного уравнения. 
    \item Обусловленность задачи решения нелинейных уравнений. Понятие об интервале неопределенности. Правило Гарвика. 
    \item Метод бисекции решения нелинейных уравнений. Скорость сходимости. Критерий окончания. 
    \item Метод Ньютона решения нелинейных уравнений. Вывод итерационной формулы метода Ньютона. 
    \item Априорная оценка погрешности метода Ньютона (теорема о скорости сходимости). 
    \item Апостериорная оценка погрешности (критерий окончания). Правило выбора начального приближения на отрезке локализации корня, гарантирующего сходимость метода. 
    \item Модификации метода Ньютона. Упрощенный метод Ньютона. Метод хорд. 
    \item Модификации метода Ньютона. Метод секущих. Скорость сходимости метода секущих. 
    \item Решение систем линейных алгебраических уравнений.  Постановка задачи. 
    \item Решение систем линейных алгебраических уравнений.  Определение понятия нормы вектора. Абсолютная и относительная погрешности вектора. 
    \item Решение систем линейных алгебраических уравнений.  Определение понятия нормы матрицы, подчиненной норме вектора. Геометрическая интерпретация нормы матрицы. 
    \item Обусловленность задачи решения системы линейных алгебраических уравнений для приближенно заданной правой части. Количественная мера обусловленности системы линейных алгебраических уравнений. Геометрическая интерпретация числа обусловленности. 
    \item Обусловленность задачи решения системы линейных алгебраических уравнений для приближенно заданных матрицы и правой части. 
    \item Метод Гаусса решения систем линейных алгебраических уравнений. Схема единственного деления. LU – разложение. Свойства метода. 
    \item Метод Гаусса решения систем линейных алгебраических уравнений. Схемы частичного и полного выбора ведущих элементов. Свойства методов. 
    \item Применение метода Гаусса к решению задач линейной алгебры. Вычисление  решений системы уравнений с несколькими правыми частями. 
    \item Применение метода Гаусса к решению задач линейной алгебры. Вычисление обратной матрицы. 
    \item Применение метода Гаусса к решению задач линейной алгебры. Вычисление выражений вида v = CWw. Вычисление определителя матрицы. 
    \item Метод Холецкого решения систем линейных алгебраических уравнений с симметричной положительно определенной матрицей. Свойства метода. 
    \item Метод прогонки решения систем линейных алгебраических уравнений с трехдиагональными матрицами. Свойства метода. 
    \item Постановка задачи приближения функций. Приближение функций обобщенными многочленами. 
    \item Приближение методом интерполяции. Интерполяция обобщенными многочленами. 
    \item Понятия линейно-независимой системы функций на заданном множестве точек. Теорема о существовании единственного решения задачи интерполяции. 
    \item Понятия ортогональной системы функций на заданном множестве точек. Утверждение о существовании единственного решения задачи интерполяции с помощью ортогональной системы функций. Решение задачи интерполяции для этого случая. 
    \item Полиномиальная интерполяция. Интерполяционный многочлен в форме Лагранжа. 
    \item Погрешность полиномиальной интерполяции. 
    \item Интерполяционный многочлен с кратными узлами. Погрешность интерполяции с кратными узлами. 
    \item Минимизация оценки погрешности интерполяции. Многочлены Чебышева и их свойства. Применение для решения задачи минимизации погрешности. 
    \item Интерполяционная формула Ньютона для неравных промежутков. Разделенные разности и их свойства. 
    \item Вывод формулы Ньютона для неравных промежутков с помощью разделенных разностей.  
    \item Интерполяционная формула Ньютона для равных промежутков. Конечные разности и их связь с разделенными разностями. 
    \item Вывод формул Ньютона для интерполирования вперед и назад. 
    \item Проблемы глобальной полиномиальной интерполяции. Интерполяция сплайнами. Определение сплайна. Интерполяционный сплайн. 
    \item Интерполяция сплайнами. Построение локального кубического интерполяционного сплайна. 
    \item Интерполяция сплайнами. Глобальные способы построения кубического интерполяционного сплайна. 
    \item Простейшие формулы численного дифференцирования. Вычисление первой производной. Погрешность формул. 
    \item Простейшие формулы численного дифференцирования. Вычисление второй производной. Погрешность формул. 
    \item Общий подход к выводу формул численного дифференцирования с помощью интерполяционного многочлена. 
    \item Обусловленность формул численного дифференцирования. 
    \item Численное интегрирование. Простейшие квадратурные формулы. Формула прямоугольников. Погрешность формулы. 
    \item Численное интегрирование. Простейшие квадратурные формулы. Формула трапеций. Погрешность формулы. 
    \item Численное интегрирование. Простейшие квадратурные формулы. Формула Симпсона. Погрешность формулы. 
    \item Апостериорные оценки погрешности квадратурных формул. Правило Рунге.
\end{enumerate}

\end{document}