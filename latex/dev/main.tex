\documentclass[14pt]{extarticle}

% Общие пакеты
\usepackage[utf8]{inputenc}
\usepackage[T2A]{fontenc}
\usepackage[russian]{babel}

% Математические пакеты
\usepackage{amsmath, amssymb, amsthm, mathrsfs}

% Пакеты для формата страницы
\usepackage[a4paper,margin=2.5cm]{geometry}
\usepackage[notlof,notlot]{tocbibind}
\usepackage{needspace}

\usepackage{titlesec, enumitem}

% Каждая секция начинается с новой страницы
\titleformat{\section}
  {\clearpage\normalfont\Large\bfseries}{\thesection}{1em}{}

% Устанавливаем отступы
\setlength{\parindent}{0pt}

\titlespacing{\subsection}{0pt}{4\baselineskip}{1\baselineskip}
\titlespacing{\subsubsection}{0pt}{2\baselineskip}{1\baselineskip}

% Отступы для текста
\clubpenalty=10000          % Запрет одиночных строк в начале страницы
\widowpenalty=10000         % Запрет одиночных строк в конце страницы
\displaywidowpenalty=10000  % То же для формул
\brokenpenalty=10000        % Запрет переносов в неподходящих местах
\tolerance=1000             % Допустимая растяжимость строк
\emergencystretch=3pt       % Экстренная растяжимость

% Минимальное количество строк для переноса абзаца
\raggedbottom              % Позволяет страницам быть разной высоты
\setlength{\parfillskip}{0pt plus 1fil}

% Настройки для предотвращения разрывов коротких абзацев
\newcommand{\needparagraph}[1]{
    \needspace{3\baselineskip}
    #1
}

% Настройки для секций (чтобы заголовки не "висели" в одиночестве)
\titlespacing{\section}{0pt}{3\baselineskip plus 1\baselineskip}{1\baselineskip}
\titlespacing{\subsection}{0pt}{4\baselineskip}{1\baselineskip}
\titlespacing{\subsubsection}{0pt}{2\baselineskip}{1\baselineskip}

% Отступы для списков
\setlist[enumerate]{noitemsep, topsep=0pt}
\setlist[itemize]{noitemsep, topsep=0pt}

\let\olditem\item
\renewcommand{\item}{\needspace{2\baselineskip}\olditem}

% Глубина нумерации
\setcounter{secnumdepth}{3}
\setcounter{tocdepth}{3}

% Ссылки
\usepackage{hyperref, cleveref}

% Цвета для блоков
\usepackage{tcolorbox, xcolor}
\definecolor{theoremcolor}{HTML}{F0F0F0}
\definecolor{bordercolor}{HTML}{708090}
\usepackage{fancyhdr}
\pagestyle{fancy}

% Устанавливаем рекомендуемую высоту заголовка
\setlength{\headheight}{17.0pt}
\addtolength{\topmargin}{-2.5pt}

\fancyhf{} % очистить все поля

% Верхний колонтитул: название и номер секции
\fancyhead[L]{\nouppercase{\leftmark}}
\fancyhead[R]{\thepage}

% Нижний колонтитул
\fancyfoot[C]{\textit{}}

\renewcommand{\headrulewidth}{0.4pt}
\renewcommand{\footrulewidth}{0.4pt}

% Подключаем tcolorbox
\tcbuselibrary{theorems,skins,breakable}

% Стиль рамки для теорем
\tcbset{theostyle/.style={
  colback=theoremcolor!50,
  colframe=bordercolor,
  fonttitle=\bfseries,
  coltitle=black,
  boxrule=0.8pt,
  arc=2mm,
  left=1.5mm,
  right=1.5mm,
  top=1mm,
  bottom=1mm,
  enhanced,
  breakable,
  before skip=\baselineskip,
  after skip=\baselineskip
}}

% Определения новых теорем
\newtcbtheorem[auto counter, number within=section]{theorem}{Теорема}{theostyle}{th}
\newtcbtheorem[auto counter, number within=section]{lemma}{Лемма}{theostyle}{lem}
\newtcbtheorem[auto counter, number within=section]{definition}{Определение}{theostyle}{def}
\newtcbtheorem[auto counter, number within=section]{example}{Пример}{theostyle}{ex}

\title{Вычмат-2025}

\begin{document}

\maketitle
\tableofcontents

\section{Основные определения}

\subsection{Предмет вычислительной математики. Метод и задачи вычислительной математики в терминах функционального анализа.}

    \subsubsection{Предмет вычислительной математики.}

        Необходимость разработки методов доведения математических исследований до числового результата привела к созданию отдельной дисциплицы - \textbf{вычислительной математики}.

        \begin{definition}{Вычислительная математика 1}{def-compmath-1}
            Область математики, которая призвана разрабатывать методы доведения до числового результата решений основных задач математического анализа, алгебры и геометрии и пути использования для этой цели современных вычислительных средств.
        \end{definition}

        \begin{definition}{Вычислительная математика-2}{def-compmath-2}
            Раздел математики, связанный с построением и анализом алгоритмов численного решения математических задач.
        \end{definition}

        Таким образом, \textbf{вычислительная математика} помогает решать численные задачи с помощью ЭВМ.

    \subsubsection{Функциональный анализ.}
        \begin{definition}{Функциональный анализ}{def-func-analysis}
            Область математики, изучающая свойства функциональных пространств.
        \end{definition}

        Для определения \textbf{задач и методов} вычислительной математики введем важнейшие \textbf{понятия функционального анализа}.

        \begin{definition}{Понятия функционального анализа}{def-func-analysis}
            \begin{itemize}
                \item Функциональные метрические пространства.
                \item Функции, определенные на функциональных пространствах.
            \end{itemize}
        \end{definition}

        \textbf{Функциональный анализ} рассматирвает элементы более общего (не евклидова) пространства.

    \subsubsection{Функциональные метрические пространства.}
        
        В функциональном анализе вместо евклидовых пространств рассматриваются абстрактные пространства, элементы которых могут иметь самую различную природу.

        \begin{definition}{Метрическое пространство}{def-metric-space}
            Абстрактное множество, для любых двух элементов $x$ и $y$ которого опрделено понятие расстояния $\rho(x, y)$.
        \end{definition}

        \begin{lemma}{Свойства расстояния}{lem-distance-features}
            Расстояние $\rho(x, y)$ должно удовлетворять следующим \textbf{свойствам:}
            \begin{enumerate}
                \item $\rho(x,y) \geq 0$, причем $\rho(x, y) = 0$ $\leftrightarrow$ $x$ совпадает с $y$.
                \item $\rho(x, y) = \rho(y, x)$.
                \item $\rho(x, y) \leq \rho(x, z) + \rho(z, y)$ $\forall$ $x, y, z \in \mathscr{R}$, где: $\mathscr{R}$ - метрическое пространство.
            \end{enumerate}
        \end{lemma}

        Евклидовы пространства с обычным определением расстояния удовлетворяют всем этим условиям. Но могут быть и другие метрические пространства.

        \begin{definition}{Пространство непрерывных функций}{def-continuous-func-space}
            Пространство $C[a, b]$ - множество всех непрерывных функций на отрезке $[a, b]$.\\
            Функция $f(x)$ непрерывная на $[a, b]$ $\leftrightarrow$ $f(x) \in C[a, b]$.
        \end{definition}

        \begin{example}{Неевлидово метрическое пространство}{ex-noneuclidean-metric-space}
            Пространства $L_{p}$, где $p \geq 1$ и $p \in \mathbb{R}$.
            $$L_{p} = \{f(x) \text{| } f(x) \in \text{C}[a, b] \text{, } \int_{a}^{b}|f(t)|^{p} \, dt < \infty\}$$

            \vspace{\baselineskip}

            Расстояние $\rho(x, y)$ в пространстве $L_{p}$ определяется следующим образом:
            $$\rho(x, y) = [\int_{a}^{b}|x(t) - y(t)|^{p}]^{\frac{1}{p}}$$
        \end{example}

        В каждом метрическом пространстве можно говорить об \textbf{окрестности данной точки}.

        \begin{definition}{Окрестность точки}{def-point-locality}
            $\varepsilon$-окрестностью точки $x$ некоторого метрического пространства $\mathscr{R}$ называется множество точек $y$ таких, что:
            $$\rho(x, y) \leq \varepsilon$$
        \end{definition}

        \begin{example}{Окрестность точки в $L_{p}$}{ex-point-locality}
            Окрестность точки в $L_{p}$ - это совокупность всех функций $y(t)$, принадлежащих $L_{p}$, для которых:
            $$\int_{a}^{b} |x(t) - y(t)|^{p} \, dt < \varepsilon^{p}$$
        \end{example}

        В вычислительной математике часто приходится заменять одну функцию $x(t)$ другой, более удобной для вычислительных целей и в каком-то смысле близкой к первой. Обычно эту вторую функцию берут из $\varepsilon$-окрестности первой.

    \subsubsection{Функции, заданные на функциональном пространстве.}
        
        \begin{definition}{Операторы функционального пространства}{def-operator}
            Пусть нам даны два абстрактных (функциональных) пространства $\mathscr{R}_{1}$ и $\mathscr{R}_{2}$. Пусть каждому элементу $x \in \mathscr{R}_{1}$ поставлен в соответствие элемент $y \in \mathscr{R}_{2}$. Тогда будем говорить, что нам задан \textbf{оператор}:\\
            $$y = A(x)$$
            с областью определения $\mathscr{R}_{1}$ и областью значений, принадлежащих $\mathscr{R}_{2}$.

            \vspace{\baselineskip}

            В частности, если $\mathscr{R}_{2}$ является областью вещественных или комплексных чисел, то оператор $A(x)$ - \textbf{функционал}.
        \end{definition}

        \begin{example}{Функционал}{ex-operator}
            Оператором (функционалом) в пространстве непрерывных функций на отрезке $[a, b]$ $C[a, b]$ - \textbf{определенный интеграл}:
            $$I(x) = \int_{a}^{b} x(t) \, dt$$
        \end{example}

    \subsubsection{Методы и задачи вычислительной математики.}

        \begin{definition}{Задачи вычислительной математики}{def-compmath-problems}
            Многие задачи в вычислительной математике могут быть записаны в виде:
            $$y = A(x)$$
            где $x$ и $y$ принадлежат заданным пространствам $\mathscr{R}_{1}$ и $\mathscr{R}_{2}$ и $A(x)$ - некоторых заданный оператор.
        \end{definition}

        Далеко не всегда с помощью средств современной математики удается точно решить эти задачи, применяя конечное число шагов. Для этого используют \textbf{методы вычислительной математики}:

        \begin{definition}{Основной метод вычислительной математики}{def-compmath-main-method}
            Замена пространств $\mathscr{R}_{1}$ и $\mathscr{R}_{1}$ и $\mathscr{R}_{2}$ и оператора $A(x)$ другими пространствами $\overline{\mathscr{R}_{1}}$ $\overline{\mathscr{R}_{2}}$ и оператором $\overline{A}$, более удобными для вичислительных целей.

            \vspace{\baselineskip}

            Иногда бывает достаточно произвести замену только пространств $\mathbb{R}_{1}$ и $\mathbb{R}_{2}$ или даже одного из них. Иногда достаточно заменить только оператор.

            \vspace{\baselineskip}

            Замена должна быть сделана так, чтобы решение новой задачи
            $$\overline{y} = \overline{A(\overline{x})}$$
            $\overline{x} \in \overline{\mathscr{R_{1}}}$, $\overline{y} \in \overline{\mathscr{R_{2}}}$, было в каком-то смысле близким к точному значению исходной задачи и его можно было бы отыскать сравнительно небольшими трудностями.\\
            Т.е.:
            $$\rho(x, \overline{x}) < \varepsilon$$
            $$\rho(y, \overline{y}) < \varepsilon$$
        \end{definition}

        \begin{example}{Применение метода}{ex-method-application}
            $f(x) \in C[a, b]$. Требуется решить задачу:
            $$y = \int_{a}^{b} f(x) \, dx$$
            причем интеграл не берется в элементарных функциях.

            \vspace{\baselineskip}

            Тогда возможны два пути:
            \begin{enumerate}
                \item \textbf{Замена пространств:} вместо $f(x)$ взять $P_{n}(x)$ - алгебраический многочлен степени $n$.
                \item \textbf{Замена оператора:} вместо интегрирования построить интегральную сумму $\sum_{i=1}^{n}f(x_{i})\Delta_{i}$.
            \end{enumerate}
        \end{example}

        Резюмируя, можно выделить \textbf{основные задачи} вычислительной математики:

        \begin{example}{Основные задачи}{ex-main-compmath-problems}
            \begin{itemize}
                \item Приближение множеств в функциональных пространствах.
                \item Приближение операторов, заданных на функциональных пространствах.
                \item Разработка рациональных алгоритмов и методов решения задач в условиях приминения современных вычислительных средств.
            \end{itemize}
        \end{example}

\subsection{Источники и классификация погрешностей результатов численного решения задач. Приближенные числа. Абсолютная и относительная погрешности. Правила записи приближенных чисел.}

    \subsubsection{Источники и классификация погрешностей}

        При решении прикладной задачи с использованием ЭВМ получить точное решение задачи практически невозможно. Получаемое решение почти всегда содержит погрешность, т.е. является приближенным. 
        \begin{definition}{Источники погрешности решения}{def-error-sources}
            Пусть y - точное значение величины, вычисление которой является целью поставленной задачи, а $y^{*}$ - ее приближенное значение:
            
            \begin{enumerate}
                \item \textbf{Неустранимая погрешность:} $\delta_{\text{н}}y^{*}$ - математическая модель и исходные данные вносят в решение ошибку, которая не может быть устранена далее.
                \item \textbf{Ошибка метода решения:} $\delta_{\text{м}}y^{*}$ - источник данной погрешности - метод решения задачи.
                \item \textbf{Вычислительная погрешность:} $\delta_{\text{в}}y^{*}$ - определяется характеристикой машины ЭВМ.
            \end{enumerate}
        \end{definition}

        Таким образом, полная погрешность результата решения задачи на ЭВМ складывается из трех составляющих:
        $$\delta y^{*} = \delta_{\text{н}}y^{*} + \delta_{\text{м}}y^{*} + \delta_{\text{в}}y^{*}$$

        \vspace{\baselineskip}

        На практике исходят из того, что погрешность метода должна быть на порядок (в 2 - 10 раз) меньше неустранимой погрешности. Желательно, чтобы величина вычислительной ошибки была хотя бы на порядок меньше величины погрешности метода.

    \subsubsection{Приближенные числа. Абсолютная и относительная погрешности.}
        
        Пусть $а$ - точное (неизвестное) значение некоторой величины, $a^{*}$ - приближенное (известное) значение той же величины (приближенное число).

        \begin{definition}{Абсолютная погрешность}{def-absolute-error}
            Модуль разности приближенного и точного значения некоторой величины:
            $$\Delta(a^{*}) = |a - a^{*}|$$
        \end{definition}

        \begin{definition}{Относительная погрешность}{def-relative-error}
            Для соотншения погрешность величины и ее значения вводят понятие \textbf{относительной погрешности}:
            $$\delta(a^{*}) = \frac{|a - a^{*}|}{|a|} = \frac{\Delta(a^{*})}{|a|}$$
        \end{definition}

        Так как значение $а$ неизвестно, то непосредственное вычисление величин $\Delta(a^{*})$ и $\delta(a^{*})$ по предыдущим формулам невозможно.

        \begin{definition}{Верхние границы погрешностей}{def-upper-limits-error}
            $\overline{\Delta(a^{*})}$ и $\overline{\delta(a^{*})}$ - верхние границы абсолютной и относительной погрешностей соответственно:
            $$|a - a^{*}| \leq \overline{\Delta(a^{*})}$$
            $$\frac{|a - a^{*}|}{|a|} \leq \overline{\delta(a^{*})}$$

            \vspace{\baselineskip}

            Причем, если величина $\overline{\Delta(a^{*})}$ известна, то:
            $$\overline{\delta(a^{*})} = \frac{\overline{\Delta(a^{*})}}{|a|}$$
        
            \vspace{\baselineskip}

            Аналогично, если известна $\overline{\delta(a^{*})}$:
            $$\overline{\Delta(a^{*})} = |a| \cdot \overline{\delta(a^{*})}$$            
        \end{definition}

    \subsubsection{Правила записи приближенных чисел.}
        
        Пусть приближенное число $a^{*}$ задано следующим образом:
        $$a^{*} = \alpha_{n}\alpha_{n-1}\ldots\alpha_{0}.\beta_{1}\beta_{2}\ldots\beta_{m}$$
        где $\alpha_{n}\alpha_{n-1}\ldots\alpha_{0}$ - целая часть, $\beta_{1}\beta_{2}\ldots\beta_{m}$ - дробная.

        \begin{definition}{Значащие цифры}{def-significant-digits}
            Все цифры в записи числа $a^{*}$, начиная с первой ненулевой слева.
        \end{definition}

        \begin{definition}{Верная цифра}{def-correct-digit}
            Значащую цифру называют \textbf{верной}, если абсолютная погрешность числа не превосходит единицы разряда, соответствующей этой цифре.
        \end{definition}

        \begin{example}{Значащие и верные цифры}{ex-significant-and-correct-digits}
            Пусть $a^{*} = 0.010300$, $\Delta(a^{*}) = 2 \cdot 10^{-6}$:
            
            \begin{enumerate}
                \item \textbf{Значащие цифры:} $10300$
                \item \textbf{Верные цифры:} $1030$
            \end{enumerate}
        \end{example}

        \begin{lemma}{Связь числа верных цифр с отностительной погрешностью}{lem-relative-correct-connection}
            Если число $a^{*}$ имеет ровно $N$ верных цифр, то $\delta(a^{*}) \sim 10^{-N}$.
        \end{lemma}

        \begin{lemma}{Правило записи}{lem-note-rule}
            Неравенство верхней границы абслолютной погрешности эквивалентно следующему:
            $$a^{*} - \overline{\Delta{a^{*}}} \leq a \leq a^{*} + \overline{\Delta{a^{*}}}$$
            
            Тот факт, что число $a^{*}$ является приближенным значением числа $a$ с абслоютной точностью $\varepsilon = \overline{\Delta(a^{*})}$ принято записывать в виде:
            $$a = a^{*} \pm \overline{\Delta(a^{*})}$$
            Как правило, числа $a^{*}$ и $\overline{\Delta(a^{*})}$ указывают с одинаковым числом цифр после десятичной точки.
        
            \vspace{\baselineskip}
        
            Аналогично из неравенства верхней границы относительной погрешности получаем:
            $$a^{*}(a - \overline{\delta{a^{*}}}) \leq a \leq a^{*}(a + \overline{\delta{a^{*}}})$$
            
            Тот факт, что число $a^{*}$ является приближенным значением числа $a$ с относительной точностью $\varepsilon = \overline{\delta(a^{*})}$ принято записывать в виде:
            $$a = a^{*}(1 \pm \overline{\delta(a^{*})})$$
        \end{lemma}

    Если число $a^{*}$ приводится в качестве результата \textbf{без указания величины погрешности}, то принято считать, что все его значащие цифры являются \textbf{верными}.


    \subsubsection{Округления.}

    \begin{definition}{Округление методом усечения}{def-truncation-rounding}
        Состоит в отбрасывании всех цифр, расположенных слева от $n$-ой значащей цифры.
    \end{definition}

    \begin{definition}{Округление по дополнению}{def-addition-rounding}
        Состоит в следующем правиле: если первая слева от отбрасываемых цифр меньше $5$, то сохраняемые цифры остаются без изменения. Иначе: в младший сохраняемый разряд добавляется единица.  
    \end{definition}

    Границы абсолютной и относительной погрешностей принято округлять в сторону увеличения.

\subsection{Погрешности арифметических операций над приближенными числами. Погрешность функции одной и многих переменных.}

    \subsubsection{Погрешности арифметических операций над приближенными числами.}
        
        \begin{theorem}{Абсолютная погрешность сложения/вычитания}{th-absolute-error-addition-subtraction}
            Абсолютная погрешность алгебраической суммы или разности не превосходит суммы абсолютных погрешностей слагаемых, т.е.:
            $$\Delta(a^{*} \pm b^{*}) \leq \Delta(a^{*}) + \Delta(b^{*})$$
        
            \begin{proof}
                $\Delta(a^{*} \pm b^{*}) = |(a \pm b) - (a^{*} \pm b^{*})| = |(a - a^{*} \pm (b - b^{*}))| \leq \Delta(a^{*}) + \Delta(b^{*})$\\
            \end{proof}
        \end{theorem}

        \begin{lemma}{Следствие теоремы абсолютной погрешности сложения/вычитания}{lem-absolute-error-addition-subtraction}
            В силу того, что $\Delta(a^{*}) \leq \overline{\Delta(a^{*})}$, получаем:
            $$\overline{\Delta(a^{*} \pm b^{*})} = \overline{\Delta(a^{*})} + \overline{\Delta(b^{*})}$$
        \end{lemma}

        \begin{theorem}{Относительная погрешность сложения/вычитания}{th-relative-error-addition-subtraction}
            Пусть a и b ненулевые числа одного знака. Тогда справедливы неравенства:
            $$\delta(a^{*} + b^{*}) \leq \delta_{\text{max}} \text{, } \delta(a^{*} - b^{*}) \leq \nu \delta_{\text{max}}$$
            где: $\delta_{\text{max}} = \text{max}\{\delta(a^{*}, \delta(b^{*}))\}$, $\nu = \frac{|a + b|}{|a - b|}$
        
            \vspace{\baselineskip}

            \begin{proof}
                $|a + b|\delta(a^{*} + b^{*}) = \Delta(a^{*} + b^{*}) \leq \Delta(a^{*}) + \Delta(b^{*}) = |a|\delta(a^{*}) + |b|\delta(b^{*}) \leq |a|\delta_{\text{max}} + |b|\delta_{\text{max}} = (|a| + |b|)\delta_{\text{max}} = |a + b|\delta_{\text{max}}$ (последний переход - равенство, т.к. числа одного знака)\\
                Т.е. $\delta(a^{*} + b^{*}) \leq \delta_{\text{max}}$

                \vspace{\baselineskip}

                $|a - b|\delta(a^{*} - b^{*}) = \Delta(a^{*} - b^{*}) \leq \Delta(a^{*} + \Delta(b^{*})) \leq |a + b| \delta_{\text{max}}$\\
                Т.е. $\delta(a^{*} - b^{*}) \leq \frac{|a + b|}{|a -b|}\delta_{\text{max}} = \nu \delta_{\text{max}}$
            \end{proof}
        \end{theorem}

        Итог: При построении численного метода решения задачи следует избегать вычитания близких чисел одного знака. Если же такое вычитание неизбежно, то следует вычислять  аргументы  с  повышенной  точностью,  учитывая  ее  потерю примерно в $\nu = \frac{|a+b|}{|a-b|}$ раз.

        \begin{theorem}{Относительная погрешность умножения/деления}{th-relative-error-mul-div}
            Для относительных погрешностей произведения и частного приближенных чисел верны оценки:
            $$\delta(a^{*}b^{*}) \leq \delta(a^{*}) + \delta(b^{*}) + \delta(a^{*})\delta(b^{*})$$
            $$\delta(\frac{a^{*}}{b^{*}}) \leq \frac{\delta(a^{*}) + \delta(b^{*})}{1 - \delta(b^{*})}$$

            \begin{proof}
                $|ab|\delta(a^{*}b^{*}) = \Delta(a^{*}b^{*}) = \Delta(a^{*}b^{*}) = |ab - a^{*}b^{*}| = |(a - a^{*})b + (b - b^{*})a - (a - a^{*})(b - b^{*})| \leq |a - a^{*}| \cdot |b| + |b - b^{*}| \cdot |a| + |a - a^{*}| \cdot |b - b^{*}| = \Delta(a^{*})|b| + \Delta(b^{*})|a| + \Delta(a^{*})\Delta(b^{*}) = c$

                \vspace{\baselineskip}

                Разделим $c$ на $|ab|$:\\
                $\frac{c}{|ab|} = \delta(a^{*}) + \delta(b^{*}) + \delta(a^{*})\delta(b^{*})$
            
                \vspace{\baselineskip}

                Итог: $\delta(a^{*}b^{*}) \leq \delta(a^{*}) + \delta(b^{*}) + \delta(a^{*})\delta(b^{*})$

                \vspace{\baselineskip}

                $|\frac{a}{b}|\delta(\frac{a^{*}}{b^{*}}) = \Delta(\frac{a^{*}}{b^{*}}) = |\frac{a}{b} - \frac{a^{*}}{b^{*}}| = |\frac{ab^{*} - a^{*}b}{bb^{*}}| = c$

                \vspace{\baselineskip}

                $|b^{*}| = |b - (b - b^{*})| = |b| \cdot |1 - \frac{b - b^{*}}{b}| \geq |b| \cdot (1 - \delta(b^{*}))$

                \vspace{\baselineskip}

                $c \leq \frac{|ab^{*} - a^{*}b|}{|b|^{2}(1 - \delta(b^{*}))}$

                \vspace{\baselineskip}

                Разделим на $|\frac{a}{b}|$:\\
                $\delta(\frac{a^{*}}{b^{*}}) \leq \frac{\delta(a^{*} + b^{*})}{1 - \delta(b^{*})}$
                
                \vspace{\baselineskip}

                Итог: $\delta(\frac{a^{*}}{b^{*}}) \leq \frac{\delta(a^{*}) + \delta(b^{*})}{1 - \delta(b^{*})}$
            \end{proof}
        \end{theorem}

        \begin{lemma}{Следствия теоремы относительная погрешность умножения/деления}{lem-relative-error-mul-div}
            Если $\delta(a^{*}) << 1$ и $\delta(b^{*}) << 1$, то для оценки границ относительных погрешностей можно использовать следующие приближенные равенства:
            $$\overline{\delta(a^{*}b^{*})} \approx \overline{\delta(a^{*})} + \overline{\delta(b^{*})}$$
            $$\overline{\delta(\frac{a^{*}}{b^{*}})} \approx \overline{\delta(a^{*})} + \overline{\delta(b^{*})}$$
        \end{lemma}

        \textbf{Общий итог:} выполнение арифметических операций над приближенными числами, как правило, сопровождается потерей точности. Единственная операция, при которой потеря не происходит, это сложение чисел одного знака. Наибольшая потеря точности может произойти при вычитании близких чисел одного знака.

    \subsubsection{Погрешность функции одной и многих переменной.}

        \begin{theorem}{Погрешность функции одной переменной}{th-error-function-one-variable}
            Формулы для границ погрешностей функции одной переменной:
            $$\overline{\Delta(y^{*})} \approx |f^{'}(x^{*})|\overline{\Delta(x^{*})}$$
            $$\overline{\delta(y^{*})} \approx \nu^{*}\overline{\delta(x^{*})}$$
            $$\overline{\delta(y^{*})} \approx \nu \overline{\delta(x^{*})}$$
        
            где $\nu^{*} = |x^{*}| \frac{f^{'}(x^{*})}{f(x^{*})}$, $\nu = |x| \frac{f^{'}(x)}{f(x)}$
        \end{theorem}

        \begin{theorem}{Погрешность функции многих переменных}{th-error-function-many-variables}
            Пусть $f(\vec{x}) = f(x_{1}, x_{2}, \ldots, x_{m})$ - дифференцируемая в области $G$ функция $m$ переменных, вычисление которой производится при приближенно заданных аргументах $x_{1}^{*}, x_{2}^{*}, \ldots, x_{m}^{*}$. Тогда:
            $$\Delta(y^{*}) \leq \sum_{j=1}^{m} \text{max}_{[x, x^{*}]} |f_{x_{j}}^{'}|\Delta(x_{j}^{*})$$
        \end{theorem}

\subsection{Корректность вычислительной задачи. Примеры корректных и некорректных задач.}

\subsection{Обусловленность вычислительной задачи. Примеры хорошо и плохо обусловленных задач.}

\subsection{Вычислительные алгоритмы. Корректность и обусловленность вычислительных алгоритмов.}

\section{Решение нелинейных уравнений, СЛАУ}

\subsection{Постановка задачи решения нелинейных уравнений. Основные этапы решения задачи.}

\subsection{Скорость сходимости итерационных методов уточнения решения нелинейного уравнения.}

\subsection{Обусловленность задачи решения нелинейных уравнений. Понятие об интервале неопределенности. Правило Гарвика.}

\section{Интерполяция}


\section{Дифференцирование и интегрирование}

\section{Список вопросов}
\begin{enumerate}
    \item Предмет вычислительной математики. Метод и задачи вычислительной математики в терминах функционального анализа. 
    \item Источники и классификация погрешностей результатов численного решения задач. Приближенные числа. Абсолютная и относительная погрешности. Правила записи приближенных чисел. 
    \item Погрешности арифметических операций над приближенными числами. Погрешность функции одной и многих переменных. 
    \item Корректность вычислительной задачи. Примеры корректных и некорректных задач. 
    \item Обусловленность вычислительной задачи. Примеры хорошо и плохо обусловленных задач. 
    \item Вычислительные алгоритмы. Корректность и обусловленность вычислительных алгоритмов. 
    \item Постановка задачи решения нелинейных уравнений. Основные этапы решения задачи. 
    \item Скорость сходимости итерационных методов уточнения решения нелинейного уравнения. 
    \item Обусловленность задачи решения нелинейных уравнений. Понятие об интервале неопределенности. Правило Гарвика. 
    \item Метод бисекции решения нелинейных уравнений. Скорость сходимости. Критерий окончания. 
    \item Метод Ньютона решения нелинейных уравнений. Вывод итерационной формулы метода Ньютона. 
    \item Априорная оценка погрешности метода Ньютона (теорема о скорости сходимости). 
    \item Апостериорная оценка погрешности (критерий окончания). Правило выбора начального приближения на отрезке локализации корня, гарантирующего сходимость метода. 
    \item Модификации метода Ньютона. Упрощенный метод Ньютона. Метод хорд. 
    \item Модификации метода Ньютона. Метод секущих. Скорость сходимости метода секущих. 
    \item Решение систем линейных алгебраических уравнений.  Постановка задачи. 
    \item Решение систем линейных алгебраических уравнений.  Определение понятия нормы вектора. Абсолютная и относительная погрешности вектора. 
    \item Решение систем линейных алгебраических уравнений.  Определение понятия нормы матрицы, подчиненной норме вектора. Геометрическая интерпретация нормы матрицы. 
    \item Обусловленность задачи решения системы линейных алгебраических уравнений для приближенно заданной правой части. Количественная мера обусловленности системы линейных алгебраических уравнений. Геометрическая интерпретация числа обусловленности. 
    \item Обусловленность задачи решения системы линейных алгебраических уравнений для приближенно заданных матрицы и правой части. 
    \item Метод Гаусса решения систем линейных алгебраических уравнений. Схема единственного деления. LU – разложение. Свойства метода. 
    \item Метод Гаусса решения систем линейных алгебраических уравнений. Схемы частичного и полного выбора ведущих элементов. Свойства методов. 
    \item Применение метода Гаусса к решению задач линейной алгебры. Вычисление  решений системы уравнений с несколькими правыми частями. 
    \item Применение метода Гаусса к решению задач линейной алгебры. Вычисление обратной матрицы. 
    \item Применение метода Гаусса к решению задач линейной алгебры. Вычисление выражений вида v = CWw. Вычисление определителя матрицы. 
    \item Метод Холецкого решения систем линейных алгебраических уравнений с симметричной положительно определенной матрицей. Свойства метода. 
    \item Метод прогонки решения систем линейных алгебраических уравнений с трехдиагональными матрицами. Свойства метода. 
    \item Постановка задачи приближения функций. Приближение функций обобщенными многочленами. 
    \item Приближение методом интерполяции. Интерполяция обобщенными многочленами. 
    \item Понятия линейно-независимой системы функций на заданном множестве точек. Теорема о существовании единственного решения задачи интерполяции. 
    \item Понятия ортогональной системы функций на заданном множестве точек. Утверждение о существовании единственного решения задачи интерполяции с помощью ортогональной системы функций. Решение задачи интерполяции для этого случая. 
    \item Полиномиальная интерполяция. Интерполяционный многочлен в форме Лагранжа. 
    \item Погрешность полиномиальной интерполяции. 
    \item Интерполяционный многочлен с кратными узлами. Погрешность интерполяции с кратными узлами. 
    \item Минимизация оценки погрешности интерполяции. Многочлены Чебышева и их свойства. Применение для решения задачи минимизации погрешности. 
    \item Интерполяционная формула Ньютона для неравных промежутков. Разделенные разности и их свойства. 
    \item Вывод формулы Ньютона для неравных промежутков с помощью разделенных разностей.  
    \item Интерполяционная формула Ньютона для равных промежутков. Конечные разности и их связь с разделенными разностями. 
    \item Вывод формул Ньютона для интерполирования вперед и назад. 
    \item Проблемы глобальной полиномиальной интерполяции. Интерполяция сплайнами. Определение сплайна. Интерполяционный сплайн. 
    \item Интерполяция сплайнами. Построение локального кубического интерполяционного сплайна. 
    \item Интерполяция сплайнами. Глобальные способы построения кубического интерполяционного сплайна. 
    \item Простейшие формулы численного дифференцирования. Вычисление первой производной. Погрешность формул. 
    \item Простейшие формулы численного дифференцирования. Вычисление второй производной. Погрешность формул. 
    \item Общий подход к выводу формул численного дифференцирования с помощью интерполяционного многочлена. 
    \item Обусловленность формул численного дифференцирования. 
    \item Численное интегрирование. Простейшие квадратурные формулы. Формула прямоугольников. Погрешность формулы. 
    \item Численное интегрирование. Простейшие квадратурные формулы. Формула трапеций. Погрешность формулы. 
    \item Численное интегрирование. Простейшие квадратурные формулы. Формула Симпсона. Погрешность формулы. 
    \item Апостериорные оценки погрешности квадратурных формул. Правило Рунге.
\end{enumerate}

\end{document}