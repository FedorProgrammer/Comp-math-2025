% Общие пакеты
\usepackage[utf8]{inputenc}
\usepackage[T2A]{fontenc}
\usepackage[russian]{babel}

% Математические пакеты
\usepackage{amsmath, amssymb, amsthm, mathrsfs}

% Пакеты для формата страницы
\usepackage[a4paper,margin=2.5cm]{geometry}
\usepackage[notlof,notlot]{tocbibind}
\usepackage{needspace}

\usepackage{titlesec, enumitem}

% Каждая секция начинается с новой страницы
\titleformat{\section}
  {\clearpage\normalfont\Large\bfseries}{\thesection}{1em}{}

% Устанавливаем отступы
\setlength{\parindent}{0pt}

\titlespacing{\subsection}{0pt}{4\baselineskip}{1\baselineskip}
\titlespacing{\subsubsection}{0pt}{2\baselineskip}{1\baselineskip}

% Отступы для текста
\clubpenalty=10000          % Запрет одиночных строк в начале страницы
\widowpenalty=10000         % Запрет одиночных строк в конце страницы
\displaywidowpenalty=10000  % То же для формул
\brokenpenalty=10000        % Запрет переносов в неподходящих местах
\tolerance=1000             % Допустимая растяжимость строк
\emergencystretch=3pt       % Экстренная растяжимость

% Минимальное количество строк для переноса абзаца
\raggedbottom              % Позволяет страницам быть разной высоты
\setlength{\parfillskip}{0pt plus 1fil}

% Настройки для предотвращения разрывов коротких абзацев
\newcommand{\needparagraph}[1]{
    \needspace{3\baselineskip}
    #1
}

% Настройки для секций (чтобы заголовки не "висели" в одиночестве)
\titlespacing{\section}{0pt}{3\baselineskip plus 1\baselineskip}{1\baselineskip}
\titlespacing{\subsection}{0pt}{4\baselineskip}{1\baselineskip}
\titlespacing{\subsubsection}{0pt}{2\baselineskip}{1\baselineskip}

% Отступы для списков
\setlist[enumerate]{noitemsep, topsep=0pt}
\setlist[itemize]{noitemsep, topsep=0pt}

\let\olditem\item
\renewcommand{\item}{\needspace{2\baselineskip}\olditem}

% Глубина нумерации
\setcounter{secnumdepth}{3}
\setcounter{tocdepth}{3}

% Ссылки
\usepackage{hyperref, cleveref}

% Цвета для блоков
\usepackage{tcolorbox, xcolor}
\definecolor{theoremcolor}{HTML}{F0F0F0}
\definecolor{bordercolor}{HTML}{708090}